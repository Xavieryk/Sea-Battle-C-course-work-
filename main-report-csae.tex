\documentclass[14pt, oneside]{altsu-report}

\worktype{Курсовая работа (2 курс)}
\title{Морской бой на языке C}
\author{Н.\,И.~Зуев}
\groupnumber{5.205-2}
\GradebookNumber{1337}
\supervisor{И.\,А.~Шмаков}
\supervisordegree{старший преподаватель}
\ministry{Министерство науки и высшего образования}
\country{Российской Федерации}
\fulluniversityname{ФГБОУ ВО Алтайский государственный университет}
\institute{Институт цифровых технологий, электроники и физики}
\department{Кафедра вычислительной техники и электроники}
\departmentchief{В.\,В.~Пашнев}
\departmentchiefdegree{к.ф.-м.н., доцент}
\shortdepartment{ВТиЭ}
\abstractRU{Пока счётчик работает не правильно! Поправьте количество рисунков и таблиц в cls-файле.}
\keysRU{Курсовая работа, морской бой, объектно-ориентированное программирование}
\keysEN{computer simulation, distributed version control}

\date{\the\year}

% Подключение файлов с библиотекой.
\addbibresource{graduate-students.bib}

% Пакет для отладки отступов.
%\usepackage{showframe}

\begin{document}
\maketitle

\setcounter{page}{2}
\makeabstract
\tableofcontents

\chapter*{Введение}
\phantomsection\addcontentsline{toc}{chapter}{ВВЕДЕНИЕ}
\textbf{Актуальность}
\newline
Актуальность данной работы заключается в изучении алгоритмов и структур данных. Работа с комплексным кодом и написание простейшего ИИ позволит неплохо развить навыки программирования и изучить парадигму объекто-ориентированного программирования. Работа над данной программой и освоение ООП в будущем поможет в решении различных задач и разработке проектов, ориентированных на ООП.
\newline
\textbf{Цель}
\newline
Разработать программу с графическим интерфейсом, с полноценным функционалом игры "Морской бой", используя язык программирования С и его графическую библиотеку SDL2.
\newline
\textbf{Задачи:}
\begin{enumerate}
\item Изучить суть и правила игры "Морской бой"
\item Изучить и освоить парадигму объекто-ориентированного прогрммирования
\item Изучить и освоить графическую библиотеку SDL2 
\item Разработать программу, с функционалом игры "Морской бой"
\end{enumerate}

% Подключение первой главы (теория):
\include{chapter-1-report-csae.tex}
% Подключение второй главы (практическая часть):
\include{chapter-2-report-csae.tex}
% Подключение третий главы (практическая часть с тестированием:
\include{chapter-3-report-csae.tex}

\chapter{Теоретичекская часть}
В этом разделе мы разберём суть работы, условия и задачи. Также изучим необходимые компоненты и теорию, необходимые для выыполнения работы.
\section{Постановка задачи}
\section{Правила и суть игры "Морской бой"}
\section{Язык программирования C}


\chapter{Практическая часть}
С помощью полученных знаний из теоретической части, мы разработаем программу "Морской бой" на языке программирования С. Реализуем стандартный режим игры, где будет стандартный размер поля и случайная генерация кораблей соперника. Добавим расширенный режим игры, его особенность заключается в том, что размеры игрового поля и количество флота задаются игроком, также на поле будут случайно разбросаны бонусы/мины. Возможность играть как против ИИ, так и против живого соперника на одном ПК.


\chapter*{Заключение}
\phantomsection\addcontentsline{toc}{chapter}{ЗАКЛЮЧЕНИЕ}

\begin{enumerate}
\item Пример ссылки на электронный источник~\cite{}.
\item Пример ссылки на книгу одного автора~\cite{book1author}.
\item Пример ссылки на книгу 5-ти и более авторов~\cite{book5author}.
\end{enumerate}

\newpage
\phantomsection\addcontentsline{toc}{chapter}{СПИСОК ИСПОЛЬЗОВАННОЙ ЛИТЕРАТУРЫ}
\printbibliography[title={Список использованной литературы}]

\appendix
\newpage
\chapter*{\raggedleft\label{appendix1}Приложение}
\phantomsection\addcontentsline{toc}{chapter}{ПРИЛОЖЕНИЕ}
%\section*{\centering\label{code:appendix}Текст программы}

\begin{center}
\label{code:appendix}Код программы "Морской бой"
\end{center}

\begin{minted}[linenos, fontsize=\footnotesize,numbersep=5pt, frame=lines, framesep=2mm]{c}

\end{minted}
\end{document}

